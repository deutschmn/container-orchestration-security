%% example text content
%% scrartcl and scrreprt starts with section, subsection, subsubsection, ...
%% scrbook starts with part (optional), chapter, section, ...
\chapter{Security Layers}

In Kubernetes there are many aspects to consider when looking at it from a security perspective. With its flexibility also comes considerable complexity. In order to come by it and discuss it in a structured manner, we have created the model depicted in Figure~\ref{fig:secmodel}. It is aimed at giving a holistic view over all aspects that might influence the security of a Kubernetes cluster and uses structure ideas from~\textcite{securingkubernetes} and~\textcite{kubernetessecurity}.

\myfig{security_model.png} %% filename in figures folder
  {width=0.6\textwidth,height=0.6\textheight}%% maximum width/height, aspect ratio will be kept
  {This security model shows the different layers that have to be considered when holistically regarding Kubernetes security.}%% caption
  {Security Model}%% optional (short) caption for table of figures
  {fig:secmodel}%% label

The model is composed of the following four layers:

\begin{enumerate}
    \item The lowest layer we call \textbf{Base Infrastructure Security} and it concerns all components on which Kubernetes itself builds. These include the Operating System, the container engine (most likely Docker), and the infrastructure of the public or private cloud provider when for example using Google Cloud Platform or Amazon Web Services as an Infrastructure as a Service (IaaS) provider. This layer is described in chapter~\ref{sec:layer1}.
    \item Upon that builds \textbf{Kubernetes Infrastructure Security} which is described in detail in chapter~\ref{sec:layer2}. It concerns everything related to Kubernetes' control-plane components and their configuration in terms of security. Potential issues there include abuse of the internal Kubernetes APIs, such as the Kubelet API, or intercepting control-plane traffic.
    \item Additionally to the configuration of the Kubernetes control-plane components themselves, there are also security components provided by Kubernetes to secure clusters. These components include Kubernetes' authentication and authorisation mechanisms, pod security policies, secrets management and many more. We group them together in the layer \textbf{Kubernetes Security Controls} as described in detail in Section~\ref{sec:layer3}.
    \item The top-most layer is called \textbf{App and Container Security} and makes use of all layers underneath it. On this layer, the actual applications run in containers, which in turn are located in pods. Exploiting of vulnerabilities in the application code might breach this layer. These topics are described in detail in Section~\ref{sec:layer4}.
\end{enumerate}

Sometimes it might not be entirely clear into which layer certain security aspects should be grouped in the model, as the boundaries can be blurry, but it still provides structure to discuss them.

The layered architecture also makes sense when considering damage control aspects of Kubernetes security: As one layer on the top breaches, the lower layers can still prevent further damage. For example, when there is a vulnerability in the application code on layer 4, a properly configured set of permissions on layer 3 for the pod might still prevent further damage. Inversely, an insecure port left open in a control plane component, such as the API server, on layer 2 might allow an attacker to infiltrate a whole cluster, taking over all pods, containers and applications on the upper layers. 
	
\section{Base Infrastructure Security} \label{sec:layer1}

% TODO research and write something meaningful

\section{Kubernetes Infrastructure Security} \label{sec:layer2}



\section{Kubernetes Security Controls} \label{sec:layer3}

Kubernetes provides cluster operates with an extensive set of options and tools to tweak the security of a cluster. These build upon the Kubernetes and base infrastructure and are very important to configure correctly. 

% TODO list the different controls

\subsection{Namespaces} \label{sec:namespaces}

As described in the documentation~\textcite{k8sdocs}, every resource in a Kubernetes cluster belongs to a namespace. All resources belonging to the control-plane components reside within the \mycode{kube-system} namespace, while by default all other objects are located in the \mycode{default} namespace. 

Generally, namespaces can be used to avoid naming conflicts, but also to allow for finer grained access control. For example, as described in Section~\ref{sec:apiAccessControl}, an RBAC Role can be created so that all pods from a particular namespace can be listed, but not for any other namespace. Use of this feature makes sense primarily when the cluster is used by a large number of users from, for example, different projects or departments.


\subsection{API Access Control} \label{sec:apiAccessControl}

% TODO ref to preliminaries where API server is explained
As explained in TODO, the only component in a Kubernetes cluster that is allowed to modify the cluster state directly is the API server. Therefore all requests that involve reading or modifying the cluster state must be performed via the Kubernetes API. It is well documented in~\textcite{k8sdocsApi} and is very powerful, as it is also used by all control-plane components to communicate. Hence it needs to be carefully protected from unauthorised access. The API server achieves that using three steps to very if and how a request should be performed, as depicted in Figure~\ref{fig:apiServerAccessControl}.

\myfig{api_access_control_itnext.png}{width=0.8\textwidth}{This image from~\textcite{securingkubernetesConfK8SClusterComponents} displays the steps performed by the API server before it persists a request made to it.}{API server access control flow}{fig:apiServerAccessControl} % TODO maybe make a picture myself

First, \textbf{Authentication} is performed to determine the identity of the user or system that is trying to access the API. After the identity of the accessing party has been determined, \textbf{Authorisation} is used to decide whether they are allowed to access or modify the requested resource. Finally \textbf{Admission Controllers} are applied to the request to validate or mutate it, before it is ultimately persisted.

\subsubsection{Authentication} \label{authentication}

Users in Kubernetes can be either service accounts or human users. The former are managed directly by Kubernetes, whereas for the latter no Kubernetes objects exist. The API server offers multiple strategies to authenticate such users. 

\begin{itemize}
    \item The default way of authentication between control plane components is using \textbf{X.509 certificates} within the public key infrastructure (PKI) of the cluster. For that to function, the API server is handed the Certificate Authority (CA) file using the start-up parameter \mycode{--client-ca-file=/path/to/ca.crt}. Users can then authenticate using a certificate signed by the provided CA. 

    % TODO to give this a bit more substance than just a mere paraphrasing of the documentation: maybe describe caveats when using certificates? 
    
    % TODO mention that there is no CRL and OSCP support see~\textcite{securingkubernetesSecComponents} (https://blog.giantswarm.io/applying-best-practice-security-controls-to-a-kubernetes-cluster/)
    
    \item For testing purposes it is possible to use a \textbf{static password file} as per the definition in the RFC 7617 "The 'Basic' HTTP Authentication Scheme" by~\textcite{RFC7617}. However, it is rarely practical in real-world use cases as the password file needs to be manually maintained.
    % TODO add other means of authentication (OpenID, Bearer token, etc.)
	
\end{itemize}

\subsubsection{Authorisation} \label{authorisation}

Authentication considers no actual permissions. It only determines the identity of the authenticating entity and several attributes about it, such as the username and group memberships. The authorisation step is when the actual mapping step of who is allowed to do what happens. 

As is the case for authentication, Kubernetes also provides several options for how authorisation can be performed as defined in the documentation~\textcite{k8sdocs}:

\begin{itemize}
    \item \textbf{Node Authorisation} is used only by \mycode{kubelet}s, so that nodes are equipped with the minimum set of permissions they need to operate within the cluster.
    \item \textbf{Attribute-based Access Control (ABAC)} grants permissions by explicit policies based on the users' attributes. ABAC, however, has been deprecated since Kubernetes version 1.6 and is not recommended to be used anymore.
    \item \textbf{Role-based Access Control (RBAC)} defines roles that come with certain permissions. These roles can then be bound to users to grant them access. RBAC is the recommended way of performing authorisation at the time of this writing.
    \item \textbf{Webhook Mode} queries an external REST service to outsource the authorisation to it.
\end{itemize}

As~\textcite{ABACvsRBAC} point out, conceptually there are advantages and disadvantages to both ABAC and RBAC. While ABAC provides more flexibility, RBAC lends itself better for analysis and risk assessment. The main advantage of ABAC is that it can perform restrictions based on users' attributes, such as their locations. In the context of Kubernetes, however, its developers~\textcite{ABACvsRBACk8s} argue that only RBAC should be used anymore. That is acceptable primarily as the third step of API access control in Kubernetes, \textit{Admission Control}, can provide filters to accommodate for such finer-grained control.

Since RBAC is the de facto standard for authorisation in modern Kubernetes clusters, in this work, we will only consider it and disregard the other options.

\myfig{rbac.png}{width=0.5\textwidth}{This graphic from~\textcite{k8sCookBook} depicts the basic building blocks of RBAC.}{RBAC building blocks}{fig:rbac}

The basic building blocks of RBAC are depicted in Figure \ref{fig:rbac}. \textit{Roles} are used to define access to certain \textit{Resources}. Rules can either cluster-wider or namespace wide, depending on the objects they control access to.

A role that allows reading pods is created by applying a YAML file like this: 

\begin{lstlisting}[frame=single]
kind: Role
apiVersion: rbac.authorization.k8s.io/v1
metadata:
  namespace: default
  name: pod-reader
rules:
- apiGroups: [""]
  resources: ["pods"]
  verbs: ["get", "watch", "list"]
\end{lstlisting}

Besides some metadata, a role defines certain API groups (in the above example the \mycode{""} indicate the core API), resources and certain verbs that correspond to the actions that are allowed on the resource. 

In order to assign the role to an \textit{Entity}, in this example case a \mycode{ServiceAccount} called \mycode{example-app-sa}, a \textit{RoleBinding} for the namespace \mycode{default} can be created by applying a YAML file like this:

\begin{lstlisting}[frame=single]
kind: RoleBinding
apiVersion: rbac.authorization.k8s.io/v1
metadata:
  name: read-pods
  namespace: default
subjects:
- kind: ServiceAccount
  name: example-app-sa 
  namespace: default
roleRef:
  kind: Role
  name: pod-reader
  apiGroup: rbac.authorization.k8s.io
\end{lstlisting}

As for all security configurations, the principle of Least Privilege applies especially when it comes to the configuration of RBAC. It has been initially defined by~\textcite{leastPrivilege} as \enquote{Every program and every privileged user of the system should operate using the least amount of privilege necessary to complete the job.}. 

Applied to Kubernetes API access control, this means that an application which does not need to access the Kubernetes API to function, as is likely the case for the vast majority of applications, should not be allowed to do so. As pointed out by \textcite{kubernetessecurity}, it is therefore recommended to disable the auto-mounting of the default service account into pods by setting \mycode{automountServiceAccountToken: false} in the pod specification. 

Section \ref{sec:???}. shows another practical example of an RBAC configuration that adheres to this principle.

% TODO ref to section where this is done


\subsubsection{Admission Control} \label{admissionControl}


\section{App and Container Security} \label{sec:layer4}

%% vim:foldmethod=expr
%% vim:fde=getline(v\:lnum)=~'^%%%%\ .\\+'?'>1'\:'='
%%% Local Variables: 
%%% mode: latex
%%% mode: auto-fill
%%% mode: flyspell
%%% eval: (ispell-change-dictionary "en_US")
%%% TeX-master: "main"
%%% End: 
